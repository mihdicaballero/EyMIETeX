% *****************************************************************************
% **************************** Custom Packages ********************************
% *************************** Graphics and figures *****************************

\usepackage{rotating} 	% Para girar imágenes
\usepackage{wrapfig} 	% Para poner texto al lado de una figura

% para usar [H] cuando incluya imágenes para forzar la posición
\usepackage{float}
\restylefloat{figure}
\usepackage{subcaption}	% Para tener subfiguras dentro de una figura

% ******************************** Tablas ************************************
\usepackage{booktabs} 	% Para un estilo más profesional
\usepackage{multirow} 	% Celdas multi fila
\usepackage{multicol} 	% Celdas multi columna
\usepackage{longtable} 	% Tablas en muchas páginas
\usepackage{tabularx} 	% Más ajustes de tablas

% **************************** Listas *********************************
% Ragged bottom evita espacios en blanco adicionales entre párrafos
\raggedbottom
% Para eliminar el exceso de espacio superior para enumeración, lista y descripción
\usepackage{enumitem}
\setlist[enumerate,itemize,description]{topsep=0em}

% *************************************************************************
% *********************** Bibliografía y Referencias ********************

\usepackage{hyperref} % Paquete para referenciar con \autoref{} que completa nombre de la referencia: Ecuación, Tabla, Figura

% Carga de estilo APA para la tesis - No modificar
\usepackage[backend=biber, style=apa,sortcites,natbib=true]{biblatex}
\input{_settings/apa}
\addbibresource{Bibliografia/bibliografia.bib} % Ubicación de bibliografia.bib a cargar, no omita la extensión .bib del nombre de archivo.
\renewcommand{\bibname}{Bibliografía}

% ********* Profundidad de índice y profundidad de numeración *************
\setcounter{secnumdepth}{2}
\setcounter{tocdepth}{2}

% ****************** Configurar modo Borrador *******************************

% Escribe draft=false para habilitar las figuras en 'borrador'
\setkeys{Gin}{draft=false}  
% Ubicación de la marca de agua. Ubicación(top/bottom)
\SetDraftWMPosition{bottom}
% Versión del borrado - por defecto es v1.0
\SetDraftVersion{v1.0}


% ******************************** Todo Notes **********************************
%% Las notas se ponen con el comando \mynote{Texto.} 
%% Solo se imprimen en el modo "draft".

\ifsetDraft
	\usepackage[colorinlistoftodos]{todonotes}
	\newcommand{\mynote}[1]{\todo[author=Nombre,size=\small,inline,color=green!40]{#1}}
\else
	\newcommand{\mynote}[1]{}
	\newcommand{\listoftodos}{}
\fi



% *************************** Highlighting Changes *****************************
%% Uncomment the following lines to be able to highlight text/modifications.
\ifsetDraft
  \usepackage{color, soul}
  \newcommand{\hlc}[2][yellow]{{\sethlcolor{#1} \hl{#2}}}
  \newcommand{\hlfix}[2]{\texthl{#1}\todo{#2}}
\else
  \newcommand{\hlc}[2]{}
  \newcommand{\hlfix}[2]{}
\fi

% Example highlight 1: \hlc{Text to be highlighted}
% Example highlight 2: \hlc[green]{Text to be highlighted in green colour}
% Example highlight 3: \hlfix{Original Text}{Fixed Text}

