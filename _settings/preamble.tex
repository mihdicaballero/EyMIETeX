% *****************************************************************************
% **************************** Preámbulo ********************************
% *************************** Figuras y gráficos *****************************

\usepackage{rotating} 	% Para girar imágenes
\usepackage{wrapfig} 	% Para poner texto al lado de una figura
\usepackage{float}		% para usar [H] cuando incluya imágenes para forzar la posición
\restylefloat{figure}
\usepackage{subcaption}	% Para tener subfiguras dentro de una figura

% ******************************** Tablas ************************************
\usepackage{booktabs} 	% Para un estilo más profesional
\usepackage{multirow} 	% Celdas multi fila
\usepackage{multicol} 	% Celdas multi columna
\usepackage{longtable} 	% Tablas en muchas páginas
\usepackage{tabularx} 	% Más ajustes de tablas

% **************************** Listas *********************************
% Ragged bottom evita espacios en blanco adicionales entre párrafos
\raggedbottom
% Para eliminar el exceso de espacio superior para enumeración, lista y descripción
\usepackage{enumitem}
\setlist[enumerate,itemize,description]{topsep=0em}

% *************************************************************************
% *********************** Bibliografía y Referencias ********************

\usepackage{hyperref} 	% Paquete para referenciar con \autoref{} que completa nombre de la referencia: Ecuación, Tabla, Figura

% Carga de estilo APA para la tesis - No modificar
\usepackage[backend=biber, style=apa,sortcites,natbib=true]{biblatex}
\DeclareLanguageMapping{spanish}{spanish-apa}  % APA en español

% ESPAÑOL - Para que se coloque "y" y no "&" como delimitador de autores con estilo APA.

\DeclareDelimFormat*{finalnamedelim}
{\ifnum\value{liststop}>2 \finalandcomma\fi\addspace\bibstring{and}\space}

% the bibliography also needs another conditional, so we can't wrap
% everything up with just the two lines above
\DeclareDelimFormat[bib,biblist]{finalnamedelim}{%
	\ifthenelse{\value{listcount}>\maxprtauth}
	{}
	{\ifthenelse{\value{liststop}>2}
		{\finalandcomma\addspace\bibstring{and}\space}
		{\addspace\bibstring{and}\space}}}

% Esta sección es para que se coloque "y" y no "&" como delimitador de autores en Referencias.
\DeclareDelimFormat*{finalnamedelim:apa:family-given}{%
	\ifthenelse{\value{listcount}>\maxprtauth}
	{}
	{\finalandcomma\addspace\bibstring{and}\space}}

% ESPAÑOL - Para que escriba "et al." en español.
\DefineBibliographyStrings{spanish}{%
	andothers = {et al.},
}
\addbibresource{Bibliografia/bibliografia.bib} % Ubicación de bibliografia.bib a cargar, no omita la extensión .bib del nombre de archivo.
\renewcommand{\bibname}{Bibliografía}
% Configurar secuencia de compilación a: PDFLaTeX - biber - PDFLaTeX - PDFLaTeX. Ver guía por más detalle.

% ********* Profundidad de índice y profundidad de numeración *************
\setcounter{secnumdepth}{2}
\setcounter{tocdepth}{2}

% ************ Configurar modo Borrador **************************
% Comentar estas líneas si no se está en modo borrador
% Escribe draft=false para habilitar las figuras en 'borrador'
%\setkeys{Gin}{draft=true}  
%% Para cambiar el texto de la marca de agua:
%\SetDraftText{Borrador}
%% Ubicación de la marca de agua. Ubicación(top/bottom)
%\SetDraftWMPosition{bottom}
%% Versión del borrado - por defecto es v1.0
%\SetDraftVersion{v1.0}

% *********************** Todo Notes **************************
%% Esto es algo opcional. Descomentar código de abajo para que funcione.
%% Las notas se ponen con el comando \mynote{Texto.} 
%% Solo se imprimen en el modo "draft".

%\ifsetDraft
%	\usepackage[colorinlistoftodos]{todonotes}\setlength{\marginparwidth}{3cm}\reversemarginpar
%	\newcommand{\mynote}[1]{\todo[author=Nombre,size=\small,inline,color=green!40]{#1}}
%\else
%	\newcommand{\mynote}[1]{}
%	\newcommand{\listoftodos}{}
%\fi

% ******************** Resaltador de cambios ************************
%% Esto es algo opcional. Descomentar código de abajo para que funcione.
%% Los cambios solo se ven en modo "draft"
% Ejemplo de resaltado 1: \hlc{Texto a resaltar}
% Ejemplo de resaltado 2: \hlc[green]{Texto a resaltar en color verde}
% Ejemplo de restalado 3: \hlfix{Texto original}{Texto modificado}

%\ifsetDraft
%  \usepackage{color, soul}
%  \newcommand{\hlc}[2][yellow]{{\sethlcolor{#1} \hl{#2}}}
%  \newcommand{\hlfix}[2]{\texthl{#1}\todo{#2}}
%\else
%  \newcommand{\hlc}[2]{}
%  \newcommand{\hlfix}[2]{}
%\fi