% ****** Template para la Especialización y Maestría en Ingeniería Estructural UTN ******
% Consulte el archivo README.md para obtener información sobre cómo utilizar la plantilla.
\documentclass[12pt,times,custombib,oneside]{EyMIETeX}

% Opciones de la clase: 
%	"draft" para activar modo borrador
% 	"print" para imprimir en papel, sin hiperreferencias.
% 	"oneside" o "twoside" para modo simple faz o doble faz
% 	"chapter" habilita solo el capítulo especificado y sus referencias.

% ******************************* Preámbulo ********************************
% *****************************************************************************
% **************************** Custom Packages ********************************
% *************************** Graphics and figures *****************************

\usepackage{rotating} 	% Para girar imágenes
\usepackage{wrapfig} 	% Para poner texto al lado de una figura

% para usar [H] cuando incluya imágenes para forzar la posición
\usepackage{float}
\restylefloat{figure}
\usepackage{subcaption}	% Para tener subfiguras dentro de una figura

% ******************************** Tablas ************************************
\usepackage{booktabs} 	% Para un estilo más profesional
\usepackage{multirow} 	% Celdas multi fila
\usepackage{multicol} 	% Celdas multi columna
\usepackage{longtable} 	% Tablas en muchas páginas
\usepackage{tabularx} 	% Más ajustes de tablas

% **************************** Listas *********************************
% Ragged bottom evita espacios en blanco adicionales entre párrafos
\raggedbottom
% Para eliminar el exceso de espacio superior para enumeración, lista y descripción
\usepackage{enumitem}
\setlist[enumerate,itemize,description]{topsep=0em}

% *************************************************************************
% *********************** Bibliografía y Referencias ********************

\usepackage{hyperref} % Paquete para referenciar con \autoref{} que completa nombre de la referencia: Ecuación, Tabla, Figura

% Carga de estilo APA para la tesis - No modificar
\usepackage[backend=biber, style=apa,sortcites,natbib=true]{biblatex}
\input{_settings/apa}
\addbibresource{Bibliografia/bibliografia.bib} % Ubicación de bibliografia.bib a cargar, no omita la extensión .bib del nombre de archivo.
\renewcommand{\bibname}{Bibliografía}

% ********* Profundidad de índice y profundidad de numeración *************
\setcounter{secnumdepth}{2}
\setcounter{tocdepth}{2}

% ****************** Configurar modo Borrador *******************************

% Escribe draft=false para habilitar las figuras en 'borrador'
\setkeys{Gin}{draft=false}  
% Ubicación de la marca de agua. Ubicación(top/bottom)
\SetDraftWMPosition{bottom}
% Versión del borrado - por defecto es v1.0
\SetDraftVersion{v1.0}


% ******************************** Todo Notes **********************************
%% Las notas se ponen con el comando \mynote{Texto.} 
%% Solo se imprimen en el modo "draft".

\ifsetDraft
	\usepackage[colorinlistoftodos]{todonotes}
	\newcommand{\mynote}[1]{\todo[author=Nombre,size=\small,inline,color=green!40]{#1}}
\else
	\newcommand{\mynote}[1]{}
	\newcommand{\listoftodos}{}
\fi



% *************************** Highlighting Changes *****************************
%% Uncomment the following lines to be able to highlight text/modifications.
\ifsetDraft
  \usepackage{color, soul}
  \newcommand{\hlc}[2][yellow]{{\sethlcolor{#1} \hl{#2}}}
  \newcommand{\hlfix}[2]{\texthl{#1}\todo{#2}}
\else
  \newcommand{\hlc}[2]{}
  \newcommand{\hlfix}[2]{}
\fi

% Example highlight 1: \hlc{Text to be highlighted}
% Example highlight 2: \hlc[green]{Text to be highlighted in green colour}
% Example highlight 3: \hlfix{Original Text}{Fixed Text}

	% Preámbulo con paquetes y comandos definidos por el usuario
% ******************* Información de tesis ********************
%% Título completo del Grado
\renewcommand{\submissiontext}{\scshape Trabajo para la obtención del título de}
\degreetitle{Especialización en Ingeniería Estructural}
%% Título del Trabajo Final Integrador
\title{Título de Especialización en Ingeniería Estructural. Conceptos y aplicaciones}
%% El nombre completo del autor.
\author{Nombre Apellido Apellido}
%% Facultad y Universidad
\university{Universidad Tecnológica Nacional}
\dept{Facultad Regional Buenos Aires}
% Logo de la universidad
\crest{\includegraphics[width=0.85\textwidth]{logo_especializacion}}

%% Director
%% Para varios directores, agregue cada director con el comando "\\"
\supervisor{Nombre Apellido Apellido}
%\supervisor{Dr. A. Nombre Apellido Apellido \\ Dr. B. Nombre Apellido Apellido}

%% Co-director (opcional, comentarlo si no hay)
%% Para varios co-directores, agregue cada director con el comando "\\"
\advisor{Nombre Apellido Apellido}
%\advisor{Dr. A. Nombre Apellido Apellido \\ Dr. B. Nombre Apellido Apellido}

%% Fecha de defensa {Mes, Año}
\degreedate{Noviembre, 2022} 			% Información de la tesis, autor y universidad.

% ************************** Modo Capítulo *******************************
% El modo de capítulo permite al usuario imprimir solo capítulos particulares con referencias. Título, Contenido, Frontmatter están deshabilitados por defecto
% Para usar, elija la opción `chapter' en la clase de documento
\ifdefineChapter
	\includeonly{Capitulos/capitulo1}
\fi

\begin{document}
	% Front matter - NO comentar
	\frontmatter
	% Creación de título
	\maketitle
	% Tabla de contenidos, lista de figuras y tablas
	\tableofcontents	
	\listoffigures
	\listoftables
	
	%  Main Matter - NO comentar
	\mainmatter
	% Capitulos
	%!TEX root = ../tesis.tex
\chapter{Introducción}

\section{Objetivo}

Poner contenido

\subsection{Ejemplo de índice}

Para resolver varios problemas de física, puede ser ventajoso expresar cualquier función arbitraria uniforme por partes como una Serie de Fourier \index{Serie de Fourier} compuesta por múltiplos de funciones seno y coseno.

El índice alfabético solo se puede ver compilando la tesis con el archivo \linebreak \verb|compilar-tesis-windows.bat| ubicado en la carpeta de la tesis.

\subsection{Ejemplo de símbolos y abreviaturas}
\begin{align}
CIF: \hspace*{5mm}F_0^j(a) = \frac{1}{2\pi \iota} \oint_{\gamma} \frac{F_0^j(z)}{z - a} dz
\end{align}

\nomenclature[z-cif]{$CIF$}{Cauchy's Integral Formula}                                % first letter Z is for Acronyms 
\nomenclature[a-F]{$F$}{complex function}                                                   % first letter A is for Roman symbols
\nomenclature[g-p]{$\pi$}{ $\simeq 3.14\ldots$}                                             % first letter G is for Greek Symbols
\nomenclature[g-i]{$\iota$}{unit imaginary number $\sqrt{-1}$}                      % first letter G is for Greek Symbols
\nomenclature[g-g]{$\gamma$}{a simply closed curve on a complex plane}  % first letter G is for Greek Symbols
\nomenclature[x-i]{$\oint_\gamma$}{integration around a curve $\gamma$} % first letter X is for Other Symbols
\nomenclature[r-j]{$j$}{superscript index}                                                       % first letter R is for superscripts
\nomenclature[s-0]{$0$}{subscript index}                                                        % first letter S is for subscripts
\nomenclature[s-crit]{crit}{Critical state}

\nomenclature[z-DEM]{DEM}{Discrete Element Method}
\nomenclature[z-FEM]{FEM}{Finite Element Method}
\nomenclature[z-PFEM]{PFEM}{Particle Finite Element Method}
\nomenclature[z-FVM]{FVM}{Finite Volume Method}
\nomenclature[z-BEM]{BEM}{Boundary Element Method}
\nomenclature[z-MPM]{MPM}{Material Point Method}
\nomenclature[z-LBM]{LBM}{Lattice Boltzmann Method}
\nomenclature[z-PCI]{PCI}{Peripheral Component Interconnect}
\nomenclature[z-USL]{USL}{Update Stress Last}
\nomenclature[z-DKT]{DKT}{Draft Kiss Tumble}
\nomenclature[z-PPC]{PPC}{Particles per cell}

La nomenclatura de símbolos y abreviaturas solo se puede ver compilando la tesis con el archivo \verb|compilar-tesis-windows.bat| ubicado en la carpeta de la tesis.

\subsection{Ejemplo de notas y cambios}
Las notas y cambios solo se pueden ver con la opción "draft" de la tesis.

Ejemplo de nota por el autor.
%\mynote{Esto es una nota de ejemplo.}

Ejemplos de subrayado y nota con cambio de texto específico.

%Ejemplo de resaltado 1: \hlc{Texto a resaltar}

%Ejemplo de resaltado 2: \hlc[green]{Texto a resaltar en color verde}

%Ejemplo destacado 3: \hlfix{Texto original}{Texto modificado}

	%!TEX root = ../tesis.tex

\chapter{Segundo capítulo}

\section{Una sección importante}

\begin{enumerate}
	\item The first topic is dull
	\item The second topic is duller
	\begin{enumerate}
		\item The first subtopic is silly
		\item The second subtopic is stupid
	\end{enumerate}
	\item The third topic is the dullest
\end{enumerate}
\begin{itemize}
	\item The first topic is dull
	\item The second topic is duller
	\begin{itemize}
		\item The first subtopic is silly
		\item The second subtopic is stupid
	\end{itemize}
	\item The third topic is the dullest
\end{itemize}

\begin{description}
\item[The first topic] is dull
\item[The second topic] is duller
\begin{description}
\item[The first subtopic] is silly
\item[The second subtopic] is stupid
\end{description}
\item[The third topic] is the dullest
\end{description}

\section{Otra sección} \label{sec:otra_seccion}

Una referencia a la \autoref{fig:plot}, dentro del \autoref{sec:otra_seccion}.

\begin{figure}[h]
	\centering
	\includegraphics[width=0.7\textwidth]{Figuras/capitulo_2/x_vs_y}
	\caption{Gráfico de ejemplo.}
	\label{fig:plot}
\end{figure}


	%!TEX root = ../tesis.tex

\chapter{Tercer capítulo}

\section{Primer sección}
Contenido de la sección \dots \footnote{Una nota al pie.}

Dentro del material bibliográfico se referencia aquí unos pocos a modo de ejemplo, estando los demás incluidos en la guía, como ser: \cite{book-example}, \cite{techreport-example} y \cite{article-example}.

\begin{table}[h]
	\caption{Una tabla mal formateada.}
	\centering
	\label{table:bad_table}
	\begin{tabular}{|l|c|c|c|c|}
	\hline & \multicolumn{2}{c}{Species I} & \multicolumn{2}{c|}{Species II} \\ 
	\hline  Dental measurement  & mean & SD  & mean & SD  \\ \hline 
	\hline	I1MD & 6.23 & 0.91 & 5.2  & 0.7  \\
	\hline 	I1LL & 7.48 & 0.56 & 8.7  & 0.71 \\
	\hline 	I2MD & 3.99 & 0.63 & 4.22 & 0.54 \\
	\hline 	I2LL & 6.81 & 0.02 & 6.66 & 0.01 \\
	\hline 	CMD & 13.47 & 0.09 & 10.55 & 0.05 \\
	\hline 	CBL & 11.88 & 0.05 & 13.11 & 0.04\\ 
	\hline 
	\end{tabular}
\end{table}

\begin{table}[h]
	\caption{Una tabla bien formateada.}
	\centering
	\label{table:good_table}
	\begin{tabular}{l c c c c}
	\toprule
	\multirow{2}{*}{Dental measurement} & \multicolumn{2}{c}{Species I} & \multicolumn{2}{c}{Species II} \\ 
	\cmidrule{2-5}  & mean & SD  & mean & SD  \\ 
	\midrule
	I1MD & 6.23 & 0.91 & 5.2  & 0.7  \\
	I1LL & 7.48 & 0.56 & 8.7  & 0.71 \\	
	I2MD & 3.99 & 0.63 & 4.22 & 0.54 \\	
	I2LL & 6.81 & 0.02 & 6.66 & 0.01 \\	
	CMD & 13.47 & 0.09 & 10.55 & 0.05 \\	
	CBL & 11.88 & 0.05 & 13.11 & 0.04\\ 
	\bottomrule
	\end{tabular}
\end{table}

		
	% Apéndices
	\begin{appendices} 
		\chapter{Primer apéndice} \label{Ape1}

Este es el primer apéndice.


		\chapter{Segundo apéndice} \label{Ape2}

Este es el segundo apéndice.


	\end{appendices}
	% Bibliografía	
	\printbibliography[heading=bibintoc]	

\end{document}
