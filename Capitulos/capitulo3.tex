%!TEX root = ../tesis.tex

\chapter{Tercer capítulo}

\section{Primer sección}
Contenido de la sección \dots \footnote{Una nota al pie.}

Dentro del material bibliográfico se referencia aquí unos pocos a modo de ejemplo, estando los demás incluidos en la guía, como ser: \cite{book-example}, \cite{techreport-example} y \cite{article-example}.

\begin{table}[h]
	\caption{Una tabla mal formateada.}
	\centering
	\label{table:bad_table}
	\begin{tabular}{|l|c|c|c|c|}
	\hline & \multicolumn{2}{c}{Species I} & \multicolumn{2}{c|}{Species II} \\ 
	\hline  Dental measurement  & mean & SD  & mean & SD  \\ \hline 
	\hline	I1MD & 6.23 & 0.91 & 5.2  & 0.7  \\
	\hline 	I1LL & 7.48 & 0.56 & 8.7  & 0.71 \\
	\hline 	I2MD & 3.99 & 0.63 & 4.22 & 0.54 \\
	\hline 	I2LL & 6.81 & 0.02 & 6.66 & 0.01 \\
	\hline 	CMD & 13.47 & 0.09 & 10.55 & 0.05 \\
	\hline 	CBL & 11.88 & 0.05 & 13.11 & 0.04\\ 
	\hline 
	\end{tabular}
\end{table}

\begin{table}[h]
	\caption{Una tabla bien formateada.}
	\centering
	\label{table:good_table}
	\begin{tabular}{l c c c c}
	\toprule
	\multirow{2}{*}{Dental measurement} & \multicolumn{2}{c}{Species I} & \multicolumn{2}{c}{Species II} \\ 
	\cmidrule{2-5}  & mean & SD  & mean & SD  \\ 
	\midrule
	I1MD & 6.23 & 0.91 & 5.2  & 0.7  \\
	I1LL & 7.48 & 0.56 & 8.7  & 0.71 \\	
	I2MD & 3.99 & 0.63 & 4.22 & 0.54 \\	
	I2LL & 6.81 & 0.02 & 6.66 & 0.01 \\	
	CMD & 13.47 & 0.09 & 10.55 & 0.05 \\	
	CBL & 11.88 & 0.05 & 13.11 & 0.04\\ 
	\bottomrule
	\end{tabular}
\end{table}
