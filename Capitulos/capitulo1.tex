%!TEX root = ../tesis.tex
\chapter{Introducción}

\section{Objetivo}

Poner contenido

\subsection{Ejemplo de índice}

Para resolver varios problemas de física, puede ser ventajoso expresar cualquier función arbitraria uniforme por partes como una Serie de Fourier \index{Serie de Fourier} compuesta por múltiplos de funciones seno y coseno.

El índice alfabético solo se puede ver compilando la tesis con el archivo \linebreak \verb|compilar-tesis-windows.bat| ubicado en la carpeta de la tesis.

\subsection{Ejemplo de símbolos y abreviaturas}
\begin{align}
CIF: \hspace*{5mm}F_0^j(a) = \frac{1}{2\pi \iota} \oint_{\gamma} \frac{F_0^j(z)}{z - a} dz
\end{align}

\nomenclature[z-cif]{$CIF$}{Cauchy's Integral Formula}                                % first letter Z is for Acronyms 
\nomenclature[a-F]{$F$}{complex function}                                                   % first letter A is for Roman symbols
\nomenclature[g-p]{$\pi$}{ $\simeq 3.14\ldots$}                                             % first letter G is for Greek Symbols
\nomenclature[g-i]{$\iota$}{unit imaginary number $\sqrt{-1}$}                      % first letter G is for Greek Symbols
\nomenclature[g-g]{$\gamma$}{a simply closed curve on a complex plane}  % first letter G is for Greek Symbols
\nomenclature[x-i]{$\oint_\gamma$}{integration around a curve $\gamma$} % first letter X is for Other Symbols
\nomenclature[r-j]{$j$}{superscript index}                                                       % first letter R is for superscripts
\nomenclature[s-0]{$0$}{subscript index}                                                        % first letter S is for subscripts
\nomenclature[s-crit]{crit}{Critical state}

\nomenclature[z-DEM]{DEM}{Discrete Element Method}
\nomenclature[z-FEM]{FEM}{Finite Element Method}
\nomenclature[z-PFEM]{PFEM}{Particle Finite Element Method}
\nomenclature[z-FVM]{FVM}{Finite Volume Method}
\nomenclature[z-BEM]{BEM}{Boundary Element Method}
\nomenclature[z-MPM]{MPM}{Material Point Method}
\nomenclature[z-LBM]{LBM}{Lattice Boltzmann Method}
\nomenclature[z-PCI]{PCI}{Peripheral Component Interconnect}
\nomenclature[z-USL]{USL}{Update Stress Last}
\nomenclature[z-DKT]{DKT}{Draft Kiss Tumble}
\nomenclature[z-PPC]{PPC}{Particles per cell}

La nomenclatura de símbolos y abreviaturas solo se puede ver compilando la tesis con el archivo \verb|compilar-tesis-windows.bat| ubicado en la carpeta de la tesis.

\subsection{Ejemplo de notas y cambios}
Las notas y cambios solo se pueden ver con la opción "draft" de la tesis.

Ejemplo de nota por el autor.
%\mynote{Esto es una nota de ejemplo.}

Ejemplos de subrayado y nota con cambio de texto específico.

%Ejemplo de resaltado 1: \hlc{Texto a resaltar}

%Ejemplo de resaltado 2: \hlc[green]{Texto a resaltar en color verde}

%Ejemplo destacado 3: \hlfix{Texto original}{Texto modificado}
